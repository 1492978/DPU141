\documentclass[16 pt]{amsart}
\usepackage{amscd,amsmath,amsthm,amssymb}
\usepackage{enumerate,varioref}
\usepackage{epsfig}
\usepackage{graphicx}
\usepackage{mathtools}
\usepackage{svg}
\newtheorem{thm}{Theorem}
\newtheorem{cor}[thm]{Corollary}
\newtheorem{lem}[thm]{Lemma}
\newtheorem{prop}[thm]{Proposition}
\theoremstyle{definition}
\newtheorem{defn}[thm]{Definition}
\theoremstyle{remark}
\newtheorem{ex}[thm]{Example}
\newtheorem{rem}[thm]{Remark}
\numberwithin{equation}{subsection}
\newcommand{\R}{\mathbb{R}}
\newcommand{\Z}{\mathbb{Z}}
\newcommand{\C}{\mathbb{C}}
\newcommand{\Q}{\mathbb{Q}}
\newcommand{\lh}{\lim_{h\rightarrow 0}}
\begin{document}

\title{Homework 2 Maths 141 Winter 2015}
\maketitle 

5.1.58. Transform by making he substitution $j = i-1$
\[
\prod_{i=n}^{2n} \frac{n-i+1}{n+i}
\]


\vspace{1in}

Solution: In the multipliers we simply need to replace all $i$ with $j+1$.  The only trick is adjusting the limits.\\

In the case that $i=n$ we have $j+1 = n$ so $j=n-1$.  In the case $i=2n$ we have $i=j+1=2n$ so $j=2n-1$.  Thus we achieve
\[
\prod\limits_{j=n-1}^{2n-1}\frac{n-(j+1)+1}{n+(j+1)}
\]



\newpage

5.2.33. Find the mistake:

Theorem: For any integer $n \geq 1$
\[
1^2 + 2^2 + \cdots + n^2 = \frac{(n)(n+1)(2n+1)}{6}.
\]

\begin{proof} by mathematical induction:

Certainly the theorem is true for $n=1$ 

because $1^1 = 1$ and $1 = 1(1+2)(2(1)+1)/6$.  So the base step is true.\\

For the inductive step, suppose for some integer $k\geq 1$

\[
k^2 = \frac{(k)(k+1)(2k+1)}{6}
\]. 
We must show that

\[
(k+1)^2 = \frac{(k+1)(k+2)(2k+3)}{6}.
\]

\end{proof}

\vspace{1in}

Solution: The inductive step is completely off base.  This won't prove anything. In this case by stating
\[
k^2 = \frac{k(k+1)(2k+1)}{6}
\]
We're specifying a very particular $k$.  The assumption should be that the sum of consecutive squares up to $k$ achieves this rather than a particular $k$.

\newpage


5.3.12:
Prove the following:


\[
\forall n\in \Z_{\geq 0}, 5| (7^n - 2^n)
\]


\vspace{1in}

Solution: We will prove this by induction:

\begin{proof}
For the base step let $n=0$.  
\[
5|(7^0-2^0)
\]
because everything divides zero.\\

For the inductive step we assume that at some point $m\geq 0$ that
\[
5|(7^m-2^m) 
\]
Let's appeal to two earlier theorems that we proved.  First
\[
a|b \wedge a|c \implies a|(b+c)
\]
Second
\[
a|b \implies a|bc
\]
Here we assume that all $a,b,c$ are integers.  So what we wish to do is build the next step from step $m$ and other basic knowledge that we have.  Notice
\[
7^{m+1} - 2^{m+1} = 7(7^m-2^m) + 2^m(7-2)
\]
(The reader should take a moment to verify that this is, in fact, true.)
We know by the inductive hypothesis that $5|(7^m-2^m)$.  By the second theorem we have stated we also know $5|7(7^m-2^m)$.  Additionally we know $5|(7-2)$ and so $5|2^m(7-2)$.  By adding we see
\[
5|7(7^m-2^m) + 2^m(7-2) 
\]
or
\[
5|(7^{m+1}-2^{m+1})
\]
Which is the desired result of the inductive step.
\end{proof}

\newpage


5.3.21. Prove the following
\[
\sqrt{n} < \frac{1}{\sqrt{1}} + \frac{1}{\sqrt{2}} + \cdots + \frac{1}{\sqrt{n}}
\]
for all $n\geq 2$.

\vspace{1in}

Solution: We again prove this by induction.  Again the base step is easy to verify.
\[
\sqrt{2} < 1/\sqrt{1}  + 1/\sqrt{2}
\]
Multiplying both sides by $\sqrt{2}$ and then subtracting 1 yields the result $1<\sqrt{2}$ which we know to be true.  So the base step holds.

Now for the inductive step we assume for some particular $m>1$ that
\[
\sqrt{m} < \frac{1}{\sqrt{1}} + \frac{1}{\sqrt{2}} + \cdots + \frac{1}{\sqrt{m}}
\]

Now we must show that
\[
\sqrt{m+1} < \frac{1}{\sqrt{1}} + \frac{1}{\sqrt{2}} + \cdots + \frac{1}{\sqrt{m+1}}
\]

We'll break this up into two smaller pieces.  First, by the inductive hypothesis we see
\[
[\sqrt{m}] + \frac{1}{\sqrt{m+1}} < \left[\frac{1}{\sqrt{1}} + \frac{1}{\sqrt{2}} + \cdots + \frac{1}{\sqrt{m}}\right] + \frac{1}{\sqrt{m+1}}
\]

If we can show that $\sqrt{m+1}$ is less than the left hand side of the inequality then by transitivity of $<$ we will achieve our result.  Now consider
\[
\sqrt{m+1} < \sqrt{m} + \frac{1}{\sqrt{m+1}} \overset{\cdot \sqrt{m+1}}{\longleftrightarrow} m+1 < \sqrt{m}\sqrt{m+1} +1
\]
Subtracting one from both sides and then dividing by $\sqrt{m}$
we see

\[
m+1 < \sqrt{m}\sqrt{m+1} +1 \overset{-1}{\longleftrightarrow} m<\sqrt{m}\sqrt{m+1}
\]

and
\[
m<\sqrt{m}\sqrt{m+1} \overset{/\sqrt{m}}{\longleftrightarrow} \sqrt{m}<\sqrt{m+1}
\]
Since $m>1$ this hold and thus we have

\[
\sqrt{m+1}<\sqrt{m}+\frac{1}{\sqrt{m+1}} < \sum_{i=1}^{m+1}\frac{1}{\sqrt{j}}
\]
which is the desired result.
\newpage

5.6.27: Prove that 
\[
F_k^2 - F_{k-1}^2= F_k F_{k-1} - F_{k+1}F_{k-1}
\] 
for all $k\geq 1$

\vspace{1in}

Solution: As written this is incorrect.  We know that $F_{k}\leq F_{k-1}$ and the only equality is $F_1=F_2=1$.
So we have
\[
0< F_k^2 - F_{k-1}^2= (F_k  - F_{k+1})F_{k-1} < 0
\]
This is an impossible inequality.

To fix this: Consider the difference of squares
\[
F_k^2 - F_{k-1}^2= (F_k  - F_{k-1})( F_{k}+F_{k-1})
\]
We know by definition that
\[
F_k + F_{k-1} = F_{k+1}
\]
So we have

\[
F_k^2 - F_{k-1}^2= (F_k  - F_{k-1})(F_{k+1}) = F_{k}F_{k+1} - F_{k-1}F_{k+1}
\]
This equality is true.
\end{document}