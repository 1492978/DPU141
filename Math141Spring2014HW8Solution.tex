\documentclass[16 pt]{amsart}
\usepackage{graphicx}
\usepackage{mathtools}
\newtheorem{thm}{Theorem}
\newtheorem{cor}[thm]{Corollary}
\newtheorem{lem}[thm]{Lemma}
\newtheorem{prop}[thm]{Proposition}
\theoremstyle{definition}
\newtheorem{defn}[thm]{Definition}
\theoremstyle{remark}
\newtheorem{ex}[thm]{Example}
\newtheorem{rem}[thm]{Remark}
\numberwithin{equation}{subsection}
\newcommand{\R}{\mathbb{R}}
\newcommand{\Z}{\mathbb{Z}}
\newcommand{\C}{\mathbb{C}}
\newcommand{\half}{\frac{1}{2}}
\newcommand{\Q}{\mathbb{Q}}
\newcommand{\lh}{\lim_{h\rightarrow 0}}
\begin{document}

\title{Homework Last Maths 141 Spring 2014}
\maketitle 

11.5.25: The recurrence relation arising from the efficieny of merge sort is given by
\[
m_n = m_{\lceil n/2 \rceil} + m_{\lfloor n/2 \rfloor} + n-1
\]
with $m_1 = 0$ since it takes no time to order a list of one element.  

Show that $m_n$ is $\Theta(n\log (n))$ by showing:\\

a. 
\[
\frac{1}{2}n\log(n) \leq m_n
\]

b.
\[
m_n \leq 2n\log(n)
\]


Solution:\\
a. Here, we will use strong induction.  First, the base step:
\[
\half 1\cdot \log(1) = 0 \leq m_1 =0.
\]
Now since we're using strong induction we'll assume
\[
\half k\log(k) \leq m_k, \forall k<n
\]
and we wish to show
\[
\half n\log(n) \leq m_n.
\]


Unfortunately for us, we have two cases to consider here, both even and odd.  Let's consider the even case first:

\begin{eqnarray*}
m_n & = & m_{n/2} + m_{n/2} + n-1\\
    & = & 2 m_{n/2} + n-1\\
    & \geq & 2(\half \frac{n}{2}\log\left(\frac{n}{2}\right)) + n-1\\
    & = & \frac{n}{2} \log\left(\frac{n}{2}\right) + n-1\\
    & = & \frac{n}{2} (\log(n)-\log(2))+n-1\\
    & = & \frac{n}{2}(\log(n)-1)+n-1\\
    & = & \frac{n}{2}\log(n) + \frac{n}{2} - 1 \\
    & \geq & \half n \log(n)
\end{eqnarray*}
This is true for all $n>1$, which is what we wanted to show.

Now for the odd case. This is mostly similar, except that the floor and ceiling functions don't evaluate directly to $n/2$.  



\begin{eqnarray*}
m_n & = & m_{(n-1)/2} + m_{(n+1)/2} + n-1\\
    & \geq & \half \frac{n-1}{2}\log\left(\frac{n-1}{2}\right) + \half \frac{n+1}{2}\log\left(\frac{n+1}{2}\right) + n-1\\
    & = & \half \frac{n-1}{2}(\log(n-1)-1)\\
    &   & + \half \frac{n+1}{2}(\log(n+1)-1)+n-1\\
    & = & \frac{n-1}{4}\log(n-1) + \frac{n+1}{4}\log(n+1) +\frac{n}{2}-1\\
    & = & \frac{1}{4}\left((n-1)\log(n-1)+(n+1)\log(n+1)\right)+ \half n-1\\
    & = & \frac{1}{4}\left((n+1-2)\log(n-1)+(n+1)\log(n+1)\right)+\half n-1\\
    & = & \frac{1}{4}\left((n+1)\log(n-1)+(n+1)\log(n+1)\right)\\
    & & -2\log(n-1)+\half n-1\\
    & \geq &  \frac{1}{4}\left(n\log(n)+n\log(n)\right)-2\log(n-1) + \half n-1\\
    & \geq & \half n\log(n)
\end{eqnarray*}
This hold true for $n>2$ but since we're in the odd case, this works.

In fact $(n+1)\log(n-1) > n\log(n)$ when $n>4$, but the additional term of $\half n$ pushes this back down to $n>2$.  



Thus we have shown
\[
m_n \text{ is } \Omega(n\log(n)).
\]
\vspace{1in}

b. Now we wish to show a similar result but we wish the show
\[
m_n \leq 2n\log(n)
\]
Again, we will argue by strong induction with the nearly trivial base step:
\[
m_1=0 \leq 2(1)\log(1)=0.
\]

Now for the strong induction we'll assume
\[
m_k \leq 2k\log(k) \forall k<n.
\]

Let's again begin with the even case.
\begin{eqnarray*}
m_n & = & m_{n/2} + m_{n/2} + n-1\\
    & = & 2 m_{n/2} + n-1\\
    & \leq & 2(2\frac{n}{2}\log\left(\frac{n}{2}\right)) + n-1\\
    & = & 2n \log\left(\frac{n}{2}\right) + n-1\\
    & = & 2n (\log(n)-\log(2))+n-1\\
    & = & 2n(\log(n)-1)+n-1\\
    & = & 2n\log(n) -n - 1 \\
    & \leq & 2n \log(n)
\end{eqnarray*}

Which is what we wished to show.

\vspace{1in}
Now let's consider the odd case

\begin{eqnarray*}
m_n & = & m_{(n-1)/2} + m_{(n+1)/2} + n-1\\
    & \leq & (n-1)\log\left(\frac{n-1}{2}\right)\\
    &      & + (n+1)\log\left(\frac{n+1}{2}\right) + n-1\\
    & = & (n-1)\log(n-1)+(n+1)\log(n+1) - n-1\\
    & = & (n-1)(\log(n-1)+\log(n+1)) + 2\log(n+1)- n-1\\
    & = & (n-1)\log(n^2-1) + 2\log(n+1) -n-1\\
    & leq & n\log(n^2)+2\log(n+1) -n-1\\
    & = & 2n\log(n) +2\log(n+1) -n - 1 \\
    & \leq & 2n \log(n)
\end{eqnarray*}
This is true for $n>2$ and since we're in the odd case, our smallest $n$ will be 3.
This is what was to be shown.\\

Thus we conclude
\[
m_n \text{ is } O(n\log(n)).
\]

Putting these together we get

\[
\half n \log(n) \leq m_n \leq 2n\log(n) 
\]
and thus
\[
m_n \text{ is } \Theta(n\log(n)).
\]
\newpage

11.5.26:  In this problem we will show that the ability to evaluate a power function $x^n$ is $O(\log(n))$ rather than the more obvious $O(n)$.\\

The main point is that we need not compute $x^n$ by simply multiplying 
\[
\underbrace{x\cdot x\cdots x}_\text{n copies}
\]
which takes $n-1$ multiplications, but rather by computing $x^n$ using the binary representation of $n$ and realizing
\[
x^{2^k} = (x^{2^{k-1}})^2
\]
SO for example, we can compute $x^{43}$ by
\[
x^{43} = x^{32+8+2+1}=x^{32}x^{8}x^2 x 
\]
Where we can compute $x^{32}$ by 5 multiplications
\[
x^{32}= (x^{16})^2= \dots = ((((x^2)^2)^2)^2)^2
\]
So in this case we compute $x^{43}$ by 12 multiplications rather than 42.\\



a. Write an algorithm to take a real number $x$ and an integer $n$ and return $x^n$ by:\\
(i) calling an algorithm to find the binary representation of $n$ and returning a sequence
\[
n_{10} = (r[k]r[k-1]\dots r[1]r[0])_2
\]
where each $r[i]$ is 0 or 1.\\


Solution (i):
Algorithm 5.1.1. is taking $n$ computing $n \mod{2}$ then $n$ div 2 and repeating until $n$ div 2 is 0.\\



Decimal-to-Binary.0
\[
\begin{array}{llll}
\text{Input: } n &&&\\
\text{Algorithm Body:} &&&\\
& q:=n, i:= 0 &&\\
& \text{while} (i=0 \text{ or } q\neq 0) &&\\
&& r[i]:= q \mod{2} &\\
&& q := q \text{ div } 2 &\\
&& i := i+1 & \\
& \text{end while} &&\\
\text{Output: } r[0]r[1]\dots r[k]
\end{array}
\]



I suggest another algorithm.  Assuming we have a good definition for $\log_2(n)$ we can call the following alternative algorithm

Decimal-to-Binary.1
\[
\begin{array}{llll}
\text{Input: } n &&&\\
\text{Algorithm Body:} &&&\\
& q:=n, k:= \lfloor \log_2(n) \rfloor &&\\
& r[k]:=1 && \\
& \text{For } i:= 1 \text{ to }k, && \\
& & r[i]:=0 & \\
&& \text{next } i & \\
& \text{end for} &&\\
& \text{while} (q\neq 0) &&\\
&& q:= q - 2^{\lfloor \log_2(q)\rfloor} &\\
&& i := \lfloor \log_2(q)\rfloor &\\
&& r[i] := 1 & \\
& \text{end while} &&\\
\text{Output: } r[0]r[1]\dots r[k]
\end{array}
\]

Here, we have initialized all $r[i]$ as zero, and we only check which of them are nonzero.\\


\vspace{1in}

(ii) Compute $x^2, x^{2^2}, \dots, x^{2^k}$ by recurssive squaring.

Solution (ii):\\
Binary-Base
\[
\begin{array}{llll}
\text{Input: } x,k &&&\\
\text{\# This k is obtained from } \lfloor \log_2(n) \rfloor&&&\\
\text{Algorithm Body:} &&&\\
& \text{For } i=1 \text{ to } k && \\
& & x^{2^i} : = (x^{2^{i-1}})^2 & \\
& & \text{next } i & \\
& \text{end for} && \\
\text{Output: } x,x^2,x^{2^2},\dots, x^{2^k} &&& \\
\end{array}
\]

\vspace{1in}

(iii) Compute  $x^n$ by using the fact that
\[
x^n = x^{r[k]2^k + \dots + r[0]2^0} = x^{r[k]2^k}\cdots x^{r[0]2^0}.
\]


Solution (iii)
Compute-Power
\[
\begin{array}{llll}
\text{Input: } x,n &&&\\
\text{Algorithm Body:} &&&\\
& k:= \lfloor \log_2(n)\rfloor &&\\
& \text{Decimal-to-Binary.0}(x,n) && \\
& \text{Binary-Base}(x,k) && \\
& y:=1 & & \\
& \text{For } i=1 \text{ to } k && \\
&& y:= y\cdot (x^{2^i})^{r[i]} & \\
&& \text{next } i &\\
&\text{end for} &&\\
\text{Output: } y &&& \\
\end{array}
\]


\vspace{1in}

b. Show that the number of multiplications used to obtain $x^n$ via the algorithm in (a) is less than or equal to $2\lfloor\log_2(n)\rfloor$\\

Solution:\\
Each $x^{2^k}$ requires $k$ multiplications (i.e. squaring $k$ times).  Each product of different $x^{2^j}$ and $x^{2^i}$ requires only one multiplication.  And so we have a maximum number of multiplications of
\[
\text{Multiplications required for } x^n \leq k + k
\]
The additional $k$ comes from the intermultiplications of the different $x^{2^j}$.

If we assume that we need to compute every $x^{2^j}$ for $j\leq k$ then we have k multiplications, and then a maximum of $k$ more for the intermultiplications between the various $x^{2^j}$ thus

\[
\text{Max}(x^n) \leq 2k = 2\lfloor \log_2(n)\rfloor 
\] 
And thus we have made multiplication $O(\log(n))$.  An exponential improvement on $O(n)$


\end{document}