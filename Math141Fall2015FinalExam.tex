\documentclass[16 pt]{amsart}
\usepackage{amscd,amsmath,amsthm,amssymb}
\usepackage{enumerate,varioref}
\usepackage{epsfig}
\usepackage{graphicx}
\usepackage{mathtools}
\newtheorem{thm}{Theorem}
\newtheorem{cor}[thm]{Corollary}
\newtheorem{lem}[thm]{Lemma}
\newtheorem{prop}[thm]{Proposition}
\theoremstyle{definition}
\newtheorem{defn}[thm]{Definition}
\theoremstyle{remark}
\newtheorem{ex}[thm]{Example}
\newtheorem{rem}[thm]{Remark}
\numberwithin{equation}{subsection}
\newcommand{\R}{\mathbb{R}}
\newcommand{\Z}{\mathbb{Z}}
\newcommand{\C}{\mathbb{C}}
\newcommand{\Q}{\mathbb{Q}}
\newcommand{\lh}{\lim_{h\rightarrow 0}}
\begin{document}

\title{Final Exam  Maths 141 Autumn 2015 \\ DePaul University\\Dr. Alexander}
\maketitle
You have 135 minutes to complete this exam.  Calculators are allowed, but no other electronic devices are permitted.  Please write all your answers in complete, legible sentences, and show all your work to receive full credit.  There are eleven (11) problems here.  Please do any ten (10) problems.
\vspace{1in}


%table
\begin{center}
  \begin{tabular}{ c | c }
    Problem & Score\\
    \hline
    &\\
    1&\\
    &\\
    2&\\
    &\\
    3&\\
    &\\
    4&\\
    &\\
    5&\\
    &\\
    6&\\
    &\\
    7&\\
    &\\
    8&\\
    &\\
    9&\\
    &\\
    10&\\
    &\\
    11& \\
    &\\
    Bonus 1&\\
    &\\
    Bonus 2&\\
    \hline 
    &\\    
    Total& 
 \end{tabular}
\end{center}

\newpage 
Problem 1. Prove the following statement by induction:
\[
\sum_{j=1}^n (3j-2) = \frac{n(3n-1)}{2}
\]

\vspace{2.5in}

Problem 2. Prove the following by induction:
\[
\begin{pmatrix}
1 & 2 \\
2 & 1
\end{pmatrix}^n =
\frac{1}{2} \begin{pmatrix}
{3^n + (-1)^n} & {3^n - (-1)^n}\\
{3^n - (-1)^n} & {3^n + (-1)^n}
\end{pmatrix}
\]


\newpage

Problem 3. Prove Demoivre's theorem:
\[
(\cos(\theta) + i\sin(\theta))^n = \cos(n\theta) + i \sin(n\theta), \text{   } \forall n \ge 0
\]
where $n$ is an integer, $i^2=-1$, and $\theta\in\R$.\\



Hint:
Recall the summation formulae: 
\[
\cos(A+B) = \cos(A)\cos(B) - \sin(A)\sin(B)
\]
\[
\sin(A+B) = \sin(A)\cos(B) + \sin(B)\cos(A)
\]

\newpage

Problem 4. Prove by induction:
\[
7 | (2^{n+2}+3^{2n+1})
\]

\newpage

Problem 5. Let $A = \Z \times \Z$ be the set of ordered pairs of integers.  Define the relation $S$ on $A$ by
\[
(a,b) S (c,d) \iff 3a+2d = 2b+3c
\]

Note: $a+d$ is an integer, but $(a,b)$ is an ordered pair of integers.\\

(a) Show that $S$ is an equivalence relation.\\

(b) Give five elements in the equivalence class $[(8,2)]$.\\

(c) What are the equivalence classes of this relation?

\newpage

Problem 6. Verify the solution for the following recurrence relation:

$\begin{array}{ccccccc}
a_{n+1} & = & \frac{5}{2} a_n &-& \frac{1}{2}b_n &+& \frac{1}{2}c_n\\
b_{n+1} & = & 3 a_n &-& b_n &-& 3 c_n\\
c_{n+1} & = & -\frac{1}{2}a_n &-& \frac{7}{2}b_n &-& \frac{1}{2}c_n
\end{array}$ \\



where $a_0=b_0=c_0=1$ is given by
\[
a_n = \frac{1}{2}(2^n+3^n)
\]
\[
b_n = \frac{1}{2}(2^n + (-4)^n)
\]
\[
c_n = \frac{1}{2}(3^n+(-4)^n)
\]
\newpage 

Problem 7. (a) Find the inverse of 73 $\mod{103}$.\\


(b) Find an integer $k$ so that $73k \equiv 3 \mod{103}$\\

Hint: After part (a) do not simply plug in numbers to find part (b).  That is inefficient, and misses the point of the question entirely.

\newpage

Problem 8. Prove the following:\\
If $(x\equiv a \mod{n})$ and $(y\equiv b\mod{n})$ then
\[
x+y \equiv a+b \mod{n}
\]
\begin{center} and \end{center}
\[
x\cdot y \equiv a\cdot b \mod{n}.
\]

\newpage

Problem 9. (a) State the definition of the class of functions $\Omega(g(x))$.\\

(b) Give the proper negation of part (a).\\

(c) Using the negation, prove
\[
n \text{ is not } \Omega(n^2)
\]

\newpage


Problem 10. Use the Euclidean algorithm to compute:
\[
\gcd(2145,210).
\]

\vspace{2.5in}

Problem 11. Show the following:
\[
n^4-n^2\log(n) \text{ is } \Theta(n^4).
\]

\newpage

Bonus 1. Prove that for all real numbers $a,b,c$ 
\[
\sum_{j=1}^{n} aj^2 + bj + c = \frac{n((a+b)(n+1)+a(2n+1)+c)}{6}
\]

In particular for any fixed $n$ there is a choice of $a,b,c$ so that the sum is zero.

\vspace{1in}

Bonus 2. The Karatsuba algorithm for multiplying large numbers is faster than the classical ``naive" multiplication algorithm (i.e. the one we learn in grade school).\\  
Classically If we wish to multiply $xy$ in base $b$ we can write (for induction purposes) 
\[
x = x_1\cdot b^m + x_0
\]
\[
y = y_1\cdot b^m + y_0
\]
And then the classical multiplication is given by:
\[
xy = z_2(b^{2m}) + z_1(b^m) + z_0
\]
where
\[
z_2 = x_1\cdot y_1, \text{ } z_1 = x_1\cdot y_0 + y_1\cdot x_0, \text{ } z_0 = x_0\cdot y_0
\]

Karatsuba showed that $z_1$ can be rewritten in terms of $z_2$ and $z_0$ as such
\[
z_1 = (x_1+x_0)\cdot(y_1+y_0) - z_2-z_0
\]
Now this requires more steps if a single multiplication is the same number of operations as a single addition.  However, we know better.  An $n$ bit addition requires $n$ elementary steps, whereas an $n$ bit multiplication requires $n^2 + n$ simple operations.  The $n^2$ are for the simple multiplications and the $n$ are for the simple column additions (think back to elementary school).\\

The advantage is that Karatsuba requires only three elementary multiplications vs the four in the classical algorithm.\\

Let $T(n)$ be the run time of Karatsuba's algorithm on two $n$ bit integers.  Prove by induction
\[
T(n) \in \Theta(n^{\log_2(3)})
\]


\end{document}
